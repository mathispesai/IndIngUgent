\documentclass[aspectratio=169,kulak,t]{kulakbeamer} % handout

\usepackage[dutch]{babel}
\usepackage[T1]{fontenc}
\usefonttheme[onlymath]{serif}

\title[Korte titel]{mijn eerste beamer-presentatie}
\author[Korte naam]{Mathis Bossuyt} 
\institute[Kulak]{KU Leuven Kulak}
\date{Academiejaar 2020 -- 2021}

\AtBeginSection[]{\only<beamer>{\addtocounter{framenumber}{-1}
		\begin{outlineframe}\frametitle{Overzicht}
			\tableofcontents[currentsection,hideallsubsections]
	\end{outlineframe}}}


\begin{document}

\begin{titleframe}
\titlepage
\end{titleframe}

\begin{outlineframe}[Overzicht]
\tableofcontents
\end{outlineframe}



\section{Inleiding}

\begin{frame}
\frametitle{Inleidende frame}
\begin{itemize}
	\item Een aantal dingen
	\item nog een aantal dingen
	\item laatste zaken
\end{itemize}
 
\end{frame}

\section{Presentatie structuren}

\begin{frame}{info foto dink}
	
	\begin{columns}
		\column{0.5\textwidth}
		\begin{figure}
			\includegraphics{ik}
			\caption{Foto van mezelf}
			
			
		\end{figure}
		
		\column{0.5\textwidth}
	
	
	
		Naam:
		Mathis Bossuyt
		
		
		Studentennummer:
		r0850546
		
		
		Richting:
		Ingenieurswetenschappen
		
		
		Favoriete formule:
		\begin{equation}
			1+e^{i\pi}=0	
		\end{equation}
		
		
		
	\end{columns}
	
	
\end{frame}	



\begin{frame}
\frametitle{blokken}
\begin{block}{definitie}
een priemgetal is een getal dat deelbaar is door 1 en zichzelf
\end{block}

\begin{exampleblock}{voorbeelden}
	voorbeelden van priemgetallen: 1, 3, 5, 7, \ldots
\end{exampleblock}

\begin{alertblock}{de Eén}
	Eén is geen priemgetal
\end{alertblock}


	
\end{frame}

\begin{frame}{kolommen}
	\begin{columns}
		\column{0.5\textwidth}
		
		Eerste kolom
		\column{0.5\textwidth}
	\begin{alertblock}{De één}
		1 is geen priemgetal
	\end{alertblock}

	\end{columns}
\end{frame}	
\begin{frame}{automatische lijstjes opbouwen}
	\begin{itemize}[<+->]
		\item dit staat er eerst
		\item Dit staat er \alert{nog} niet
		\item dit stuk wordt \only{helemaal niet} getoond
		\item dit stuk wordt \uncover{niet} getoond.
		\item dit stuk wordt \uncover{niet} getoond.
	\end{itemize}
\hypertarget<3>{auto:3}{Zichtbaar}
\hypertarget<4>{auto:4}{}
\end{frame}	

\begin{frame}{hyperspace}

\hyperlink{auto:3}{klik hier voor slide 3}

\hyperlink{auto:4}{klik hier voor slide 4}

\end{frame}
\section{Besluit}
\begin{frame}
\frametitle{Afsluitende frame}
Afsluitende tekst.
\end{frame}

\end{document}
